\mainsection{1}{Lecture Title}{dd/mm/yyyy}

%%%%%%%%%%%%%%%%%%%%%%%%%%%
% 	You can use the custom putfigure command to add figures to the document. the command takes 6 arguments - how much horizontal size to create a bounding box, width, height, path to image, caption, label.

% \putfigure{1.0}{0.7}{0.3}{Images/rl1}{Many Faces of RL}{fig1_1}

%%%%%%%%%%%%%%%%%%%%%%%%%%%
% Adding multiple images using putfigure with minipage

% \begin{minipage}{\textwidth}
% 	\centering
% 		\putfigure{0.49}{1.0}{0.2}{Images/rl2}{A generic RL schematic diagram}{fig1_2}
% 		\putfigure{0.49}{1.0}{0.2}{Images/rl3}{An RL agent with environment loop}{fig1_3}
% \end{minipage}

%%%%%%%%%%%%%%%%%%%%%%%%%%%
% You can use the myequations command to add any equation to the list of equations in contents. Just pass the name for that equation with the command:

% \verb!\begin{align}! \\
% \verb!    \S_t = \f(\H_t)! \\
% \verb!\end{align}! \\
% \verb!\myequations{Generic state RL system}! \\
%%%%%%%%%%%%%%%%%%%%%%%%%%%
	

\section{Title 1}

\begin{align}
    \S_t = \f(\H_t)
\end{align}
\myequations{Generic state RL system}